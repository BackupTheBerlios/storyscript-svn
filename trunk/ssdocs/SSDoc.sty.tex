%
% Copyright (c) 2004-2005 Daniel Jones % (DanielCJones@gmail.com)
% 
% This is part of the  StoryScript (AKA: SS, S^2, % SSqared, etc) software.  Full license information % is included in the file in the top directory % named "license".
% 
% NOTES: Contains declarations for the Anomaly class,
% 	which serves as the base exception type.
% 



% I really really don't know what I'm doing here


\newcommand{\SSquared}{$S^{2}$}

%\newcommand{\SSquaredInterpreter}{S$^2_i$}


% SYNTAX HIGHLIGHTING STUFF

%\usepackage[a4paper,margin=2cm]{geometry}
%\usepackage[T1]{fontenc}
\usepackage{color}
%\usepackage{alltt}
%\usepackage{times}
%\usepackage{palatino}
\usepackage{newcent}
%\usepackage{pslatex}
%\usepackage{mathptmx}
%\usepackage{mathpazo}
\usepackage{anysize}
\usepackage{appendix}
\usepackage{graphicx}
%\usepackage{psfrag}

%PDF STUFF
%\usepackage[pdftex,colorlinks=true]{hyperref}

%\pdfinfo{
%	\Title (The SS Programming Language)
%	\Author (Daniel C. Jones)
%}

%\hypersetup{%
%	pdftitle={The SS Programming Language}
%	pdfauthor={Daniel C. Jones},
%	bookmarksnumbered,
%	pdfstartview={FitH},
%	urlcolor=cyan,
%}%

% For emphasizing SS Code inside english.
\newcommand{\SSCode}[1]{\begin{samepage}\texttt{#1}\end{samepage}}

%TABLES
\newcommand{\BeginSSTable}[2]{
	\begin{table}[#2]
	\begin{center}
	\begin{tabular}{#1}
	\hline
}

\newcommand{\EndSSTable}[1]{
	\hline
	\end{tabular}
	\caption{#1}
	\end{center}
	\end{table}
}


%Name, Inputs, Outputs, Comments
\newcommand{\SSFunctionDesc}[4]{
\subsection{#1}
\textbf{Inputs:} #2\\
\textbf{Effects:} #3\\
\textbf{Comments:} #4
}

%Name, Evaluation, Comments
\newcommand{\SSVariableDesc}[3]{
\subsection{#1 (\emph{Magic Variable)}}
\textbf{Evaluation:} #2\\
\textbf{Comments:} #3
}


\marginsize{1.25in}{1.75in}{1.25in}{1.25in}

% For numbering code segments and explaining them
\newcommand{\SSCodeNumber}[1]{\ensuremath{\mathbin{\settowidth{\dimen7}{\mbox{$\bigcirc$}}%
              \makebox[0pt][l]{$\bigcirc$}\makebox[\dimen7]{#1}}}}


\definecolor{SSCodeBoxColor}{rgb}{0.95,0.95,0.95}

% Ripped from some latex tutorial.
%\newlength\Linewidth
%\def\findlength{\setlength\Linewidth\linewidth
%\addtolength\Linewidth{-4\fboxrule}
%\addtolength\Linewidth{-3\fboxsep}
%}


%\newcommand\SSCodeSeperator{
%	\begin{center}
%	\noindent\rule{250pt}{.5pt}% \\
%	%\rule[-.5pt]{.5pt}{10pt}\rule{249pt}{0pt}\rule{.5pt}{10pt}
%	\end{center}
%}

\newenvironment{SSCodeBox}
{
%	\vspace{10pt}
%	\scriptsize{Code:}
%	\SSCodeSeperator
\begin{quotation}
\nopagebreak
\noindent}
{
\nopagebreak
\end{quotation}
%	\SSCodeSeperator
\normalcolor\par\addvspace{6pt minus 3pt}
}

% I have to manually define the colors

\definecolor{SSCodeBack}{rgb}{0.0,0.0,0.0}
\definecolor{SSCommentColor}{rgb}{0.9,0.9,0.9}

% Default
\definecolor{scitea}{rgb}{0.0,0.0,0.0}
\newcommand{\scitea}[1]{\noindent{\ttfamily{\textcolor{scitea}{#1}}}}

% String
\definecolor{sciteb}{rgb}{0.5,0.0,0.0}
\newcommand{\sciteb}[1]{{\ttfamily{\textbf{\textcolor{sciteb}{#1}}}}}

% Number
\definecolor{scitec}{rgb}{0.5,0.0,0.0}
\newcommand{\scitec}[1]{{\ttfamily{\textcolor{scitec}{#1}}}}

% Keyword
\definecolor{scited}{rgb}{0.0,0.0,0.5}
\newcommand{\scited}[1]{{\ttfamily{\textcolor{scited}{#1}}}}

% Control
\definecolor{scitee}{rgb}{0.0,0.0,0.5}
\newcommand{\scitee}[1]{\ttfamily{\textbf{\textcolor{scitee}{#1}}}}

% Character Comment (line)
\definecolor{scitef}{rgb}{0.5,0.2,0.0}
\newcommand{\scitef}[1]{{\ttfamily{\textit{\textcolor{scitef}{#1}}}}}

% Utility Comment (line)
\definecolor{sciteg}{rgb}{0.5,0.5,0.5}
\newcommand{\sciteg}[1]{{\ttfamily{\textit{\textcolor{sciteg}{#1}}}}}

% Scene Comment (line)
\definecolor{sciteh}{rgb}{0.2,0.5,0.0}
\newcommand{\sciteh}[1]{{\ttfamily{\textit{\textcolor{sciteh}{#1}}}}}

% Character Comment (block)
\definecolor{scitei}{rgb}{0.5,0.2,0.0}
\newcommand{\scitei}[1]{{\ttfamily{\textit{\textcolor{scitei}{#1}}}}}

% Scene Comment (block)
\definecolor{scitej}{rgb}{0.2,0.5,0.0}
\newcommand{\scitej}[1]{{\ttfamily{\textit{\textcolor{scitej}{#1}}}}}

% Utility Comment (block)
\definecolor{scitek}{rgb}{0.5,0.5,0.5}
\newcommand{\scitek}[1]{{\ttfamily{\textit{\textcolor{scitek}{#1}}}}}

% Header
\definecolor{scitel}{rgb}{0.0,0.0,0.0}
\newcommand{\scitel}[1]{{\ttfamily{\textbf{\textcolor{scitel}{#1}}}}}

% Out-String
\definecolor{scitem}{rgb}{0.5,0.0,0.0}
\newcommand{\scitem}[1]{{\ttfamily{\textbf{\textcolor{scitem}{#1}}}}}

\definecolor{sciten}{rgb}{0.2,0.5,0.0}
\newcommand{\sciten}[1]{{\ttfamily{\textbf{\textcolor{sciten}{#1}}}}}

